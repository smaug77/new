../../template.tex

\begin{document}

\printanswers

\title{Erlang I Homework 4}
\maketitle

This section has \numpoints\ points.

\begin{questions}
    \question (\totalpoints\ points) What is the result of the
    following entered into the erlang shell?
    \begin{parts} 
        \part[1]{\tt 11<ten.}  
        \begin{solution}
            {\tt true}
        \end{solution}
        \part[1] {\tt \{123, 345\} < [].}
        \begin{solution}
            {\tt true}
        \end{solution}
        \part[1] {\tt [boo, hoo] < [adder, zebra, bee]. }
        \begin{solution}
            {\tt false}
        \end{solution}
        \part[1] {\tt [boo, hoo] < [boo, hoo, adder, zebra, bee].}
        \begin{solution}
            {\tt true}
        \end{solution}
        \part[1] {\tt \{boo,hoo\} < \{adder,zebra,bee\}.}
        \begin{solution}
            {\tt true}
        \end{solution}
        \part[1] {\tt \{boo, hoo\} < \{boo, hoo, adder, zebra, bee\}.}
        \begin{solution}
            {\tt true}
        \end{solution}
        \part[1] {\tt 1.0 == 1.}
        \begin{solution}
            {\tt true}
        \end{solution}
        \part[1] {\tt 1.0 =:= 1.}
        \begin{solution}
            {\tt false}
        \end{solution}
        \part[1] {\tt \{1,2\} < [1,2].}
        \begin{solution}
            {\tt true}
        \end{solution}
        \part[1] {\tt 1 =< 1.2.}
        \begin{solution}
            {\tt true}
        \end{solution}
        \part[1] {\tt 1 =/= 1.0.}
        \begin{solution}
            {\tt true}
        \end{solution}
        \part[1] {\tt (1<2) < 3.}
        \begin{solution}
            {\tt false}
        \end{solution}
        \part[1] {\tt (1 > 2) == false.}
        \begin{solution}
            {\tt true}
        \end{solution}
    \end{parts}

    \question[4] Which of the following are variable names, atoms or
    neither?

    \begin{enumerate}
    \item {\tt A\_long\_variable\_name}
    \item {\tt Flag}
    \item {\tt january}
    \item {\tt Name2}
    \item {\tt fooBar}
    \item {\tt DbgFlag}
    \item {\tt node@ramone}
    \item {\tt Node@Ramone}
    \item {\tt Double}
    \item {\tt NewDouble}
    \item {\tt alfa21}
    \item {\tt Happy\_days2}
    \item {\tt happy.days2}
    \item {\tt Happy.Days2}
    \item {\tt starts\_with\_lower\_case}
    \end{enumerate}

    \begin{solution}
    Variables:
    \begin{enumerate}
    \item {\tt A\_long\_variable\_name}
    \item {\tt Flag}
    \item {\tt Name2}
    \item {\tt DbgFlag}
    \item {\tt Double}
    \item {\tt NewDouble}
    \item {\tt Happy\_days2}
    \end{enumerate}

    Atoms:
    \begin{enumerate}
    \item {\tt january}
    \item {\tt fooBar}
    \item {\tt node@ramone}
    \item {\tt alfa21}
    \item {\tt happy.days2}
    \item {\tt starts\_with\_lower\_case}
    \end{enumerate}

    Neither:
    \begin{enumerate}
    \item {\tt Node@Ramone}
    \item {\tt Happy.Days2}    
    \end{enumerate}
    \end{solution}

    \question[8] Create a data structure to store information about
    people. One is Joe Armstrong, shoe size 42 with two cats - zorro
    and daisy - and two children - Thomas (21) and Claire (17). The
    other is Mike WIlliams, shoe size 41 who likes boats and
    wine. Then create a structure to store these two people.

    \begin{solution}
    {\tt 

    JoeAttributeList = [{shoeSize, 42}, {pets, [{cat, zorro},
    {cat, daisy}]}, {children, [{thomas,21},{claire,17}]}].
   
    JoeTuple = {person, ``Joe'', ``Armstrong'', JoeAttributeList}.

    MikeAttributeList = [{showSize, 42},{likes,[boats,wine]}].

    MikeTuple = {person, ``Mike'', ``Williams'', MikeAttributeList}.

    People = [JoeTuple, MikeTuple].
    }
    \end{solution}

    \question (\totalpoints\ points) Consider the following module:
{\tt
\begin{verbatim}
-module(demo).
-export([double/1]).

% This is a comment.

double(Value) ->
  times(Value, 2).
times(X,Y) ->
  X*Y.
\end{verbatim}
}
    \begin{parts}
    \part[1] How would you compile this?
    \begin{solution}
        In the erlang shell type {\tt c(demo)}.
    \end{solution}
    \part[1] What happens when you call {\tt demo:times(3,5). }?
    \begin{solution}
        {\tt 15}
    \end{solution}
    \part[1] What happens when you call {\tt double(6). }?
    \begin{solution}
        An error since the double function cannot be found.
    \end{solution}
    \end{parts}

    \question[10] Write a module {\tt shapes} that
    contains one function - {\tt area}. This area function should work
    on squares, circle, triangles and returns an error for other
    shapes. If the three lengths of a triangle you may want to use the
    formula area = $S \cdot (S-A) \cdot (S-B) \cdot (S-C)$ where $A,
    B, C$ are the length of the three sides and $S = \sqrt{ (A+B+C)/2
    }$. Be sure to compile this function and test that it works.
    \begin{solution}
    shapes.erl:
{\tt
\begin{verbatim}
-module(shapes).
-export([area/1]).

area({circle, Radius}) ->
  math:pi()*Radius*Radius;
area({square, Side}) ->
  Side*Side;
area({triangle, A, B, C}) ->
  S = (A+B+C)/2,
  math:sqrt(S*(S-A)*(S-B)*(S-C));
area(_) ->
  {error, "Unknown Shape"}.
\end{verbatim}
}

    To compile use {\tt c(shapes).}

    test\_shapes.erl:
{\tt
\begin{verbatim}
shapes:area({circle, 1}).
shapes:area({triangle, 1, 1, math:sqrt(2)}).
shapes:area({square, 2}).
shapes:area({rectangle, 4, 2}).
\end{verbatim}
}

    Running these tests get:
{\tt
\begin{verbatim}
bash-3.2$ erl < tests_shapes.erl 
Eshell V5.9  (abort with ^G)
1> 3.141592653589793
2> 0.49999999999999983
3> 4
4> {error,"Unknown Shape"}
5> *** Terminating erlang (nonode@nohost)
\end{verbatim}
}
    \end{solution}
 
    \question[15] Write a module {\tt boolean.erl} that takes logical
    expressions and Boolean values (represented as the atoms {\tt
    true} and {\tt false}) and returns their Boolean results. The
    functions you write should include {\tt b\_not/1, b\_and/2, b\_or/2},
    and {\tt b\_and/2}. You should not use the logical constructs {\tt
    and, or} but instead use pattern matching to achieve your goal. Be
    sure to test your module. For example:

    {\tt
    $bool:b\_not(false) \to true \\
    bool:b\_and(false, true) \to false \\
    bool:b\_and(bool:b\_not(bool:b\_and(true, false)), true) \to true \\
    $}

    Hint: implement {\tt b\_nand/2} using {\tt b\_not/1} and {\tt
    b\_and/2}.  

    \begin{solution}
    {\tt
\begin{verbatim}
bash-3.2$ cat bool.erl 
-module(bool).
-export([b_not/1, b_and/2, b_or/2, b_nand/2]).

b_not(true) ->
    false;
b_not(false) ->
    true;
b_not(Other) ->
    {error, "Must evaluate to atoms true or false"}.

b_and(true, true) ->
    true;
b_and(true, false) ->
    false;
b_and(false, true) ->
    false;
b_and(false, false) ->
    false;
b_and(Other, Other2) ->
    {error, "two arguments must be boolean atoms"}.

b_or(true, true) ->
    true;
b_or(true, false) ->
    true;
b_or(false, true) ->
    true;
b_or(false, false) ->
    false;
b_or(Other, Other2) ->
    {error, "two arguments must be boolean atoms"}.

b_nand(X, Y) ->
    b_not(b_and(X, Y)).

bash-3.2$ cat test_boolean.erl 
bool:b_not(false).
bool:b_and(false, true).
bool:b_and(true, bool:b_not(bool:b_and(true, false))).
bool:b_nand(false, false).
bool:b_nand(true, true).
bool:b_or(bool:b_or(false, false), true).

bash-3.2$ erl < test_boolean.erl 
Eshell V5.9  (abort with ^G)
1> true
2> false
3> true
4> true
5> false
6> true
7> *** Terminating erlang (nonode@nohost)
\end{verbatim}
        }
        
\end{solution}
\end{questions}

\end{document}
