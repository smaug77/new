../../template.tex

\begin{document}

%\printanswers

\title{575 Homework 1}
\maketitle

This section has \numpoints\ points.

\section{Lecture 1}


\begin{questions}
    \question[5] Prove that, for all positive integers $n$,
        $$1^3 + 2^3 + \cdots + n^3 = \frac{n^2(n+1)^2}{4}.$$
    \begin{solution}
        Base case: $1^3 = \frac{1^2(2)^2}{4} = 1$.

        Induction Step: 
        $$\sum_{k=1}^{n+1} k^3 = (n+1)^3 + \sum_{k=1}^n k^3$$
        $$  (n+1)^3 + \frac{n^2(n+1)^2}{4}
          = \frac{ 4(n+1)^3 + n^2(n+1)^2 }{4}$$
        $$ = \frac{ (n+1)^2 ( 4(n+1) + n^2 ) }{4}
           = \frac{ (n+1)^2 (n+2)^2 }{4}.$$        
    \end{solution}

    \question (\totalpoints\ points)
        \begin{parts}
            \part[10] Use mathematical induction to show that
                $$\frac{1}{1^2} + \frac{1}{2^2} + \frac{1}{3^2} + \cdots +
                  \frac{1}{n^2} \le 2 - \frac{1}{n}.$$
                Note how a straightforward induction on $n$ would not suffice
                to prove a {\it weaker} claim that $\frac{1}{1^2} + 
                \frac{1}{2^2} + \frac{1}{3^2}
                  + \cdots+ \frac{1}{n^2} \le 2$. It can be shown that
                  this sum is equal to $\frac{\pi^2}{6} \approx 1.6449.$
            \begin{solution}
               Base Case: 
               $$\frac{1}{1^2} = 1 = 2 - \frac{1}{1}$$

               Induction Step:
               $$\sum_{k=1}^n \frac{1}{n^2} \le 2 - \frac{1}{n}$$
               $$\sum_{k=1}^n + \frac{1}{(n+1)^2}\le 2 - \frac{1}{n}
                 + \frac{1}{(n+1)^2}$$
               $$ = 2 + \frac{-(n+1)^2 + n}{n(n+1)^2}
                  = 2 + \frac{-n^2 - n -1}{n(n+1)^2}$$
               $$ \le 2 + \frac{-n^2 - n}{n(n+1)^2}
                   = 2 - \frac{n(n+1)}{n(n+1)^2}$$
               $$ = 2 - \frac{1}{n+1}.$$
            \end{solution}
            \part[20] Use mathematical induction to prove that 
                $$\frac{1}{1^2} + \frac{1}{2^2} + \frac{1}{3^2} + \cdots +
                  \frac{1}{n^2} \le 1.674.$$
            \begin{solution}
                We prove that the sum is less than $1.674
                  - \frac{1}{n}$ then taking limits we get the result.

               Base case:
               Note that for $n = 4$:
               $$1 + \frac{1}{4} + \frac{1}{9} + \frac{1}{16} \le
               1.674 - \frac{1}{4}.$$1               

               Induction Step:
               $$\sum_{k=1}^n \frac{1}{n^2} \le 1.674 - \frac{1}{n}$$
               $$\sum_{k=1}^n + \frac{1}{(n+1)^2}\le 1.674 - \frac{1}{n}
                 + \frac{1}{(n+1)^2}$$
               $$ = 1.674 + \frac{-(n+1)^2 + n}{n(n+1)^2}
                  = 1.674 + \frac{-n^2 - n -1}{n(n+1)^2}$$
               $$ \le 1.674 + \frac{-n^2 - n}{n(n+1)^2}
                   = 1.674 - \frac{n(n+1)}{n(n+1)^2}$$
               $$ = 1.674 - \frac{1}{n+1}.$$
                                
            \end{solution}
        \end{parts}

    \question  (\totalpoints\ points)

        \begin{parts}
            \part[5] What is wrong with the following ``proof'' by induction
                  that all fire engines are the same color (e.g. red)?

                  {\bf ``Proof.''} We prove that any $n$ fire engines
                  are the same color by induction on $n$. For $n = 1$,
                  there is one fire engine and the claim is
                  obvious. Now for the inductive step. Suppose that
                  any $n$ fire engiens are the same and let us prove
                  that any $n + 1$ fire engines are the same
                  color. Well, let us arrange the $n + 1$ fire engines
                  in a straight line. By induction hypothesis, the
                  first $n$ fire engines are the same color, and so
                  are the last $n$. It follows that all fire engines
                  are the same color as the middle ones, hence all $n
                  + 1$ fire engines are the same color. This completes
                  the induction step.

            \begin{solution}
                There may be no middle. When $n$ equals two the first 1 and
                last 1 means no middle.
            \end{solution}

            \part[10] What follows is an alleged proof of the following
            theorem: ``Let $A$ be an $n \times n$ matrix whose all
            entries are non-negative integers, such that for any zero
            entry, the sum of entries in the row and the column
            containing that zero is at least $n$. Then the sum of all
            elements $A$ is at least $n^2/2$.'' What is wrong with it?

            {\bf ``Proof.''} The result holds for $n = 1$. Assume it
            holds for $(n - 1) \times (n - 1)$ matrices and let $A$ be
            an $n \times n$ matrix as described. If there are no zero
            entries, the result is clear. Suppose that $a_{ij} = 0$
            for some $1 \le i, j \le n$. The sum of elements in the
            $i$th row and the $j$th column is at least $n$. By the
            induction hypothesis the sum of the entries in the
            remaining $(n - 1) \times (n - 1)$ submatrix is at least
            $(n-1)^2/2$. And so the sum of all entries of $A$ is at
            least $$\frac{(n-1)^2}{2} + n = \frac{n^2 + 1}{2}
            > \frac{n^2}{2},$$ completing the induction.

            \begin{solution}
                We don't know that the we can apply the induction
                hypothesis to the submatrix. There may not be even a
                zero entry!
            \end{solution}
        \end{parts}

    \question  (\totalpoints\ points)
        \begin{parts}
            \part[10] Show that $\sqrt{6}$ is irrational using infinite
                  descent.
            \begin{solution}
                Suppose that $\sqrt{6}$ were rational. Then it could
                be written as $$\sqrt{6} = \frac{p}{q}$$ for two
                natural numbers, $p$ and $q$. Then squaring would give
                $$6 = \frac{p^2}{q^2}.$$ $$6q^2 = p^2,$$ so $$6 |
                p^2.$$ This implies that $$6 | p,$$ which means that 6
                divides $p$. 

                Thus we can write $p = 6r$, thus $$6q^2 = (6r)^2 =
                36r^2,$$ $$q^2 = 6r^2,$$ so $$6 | q^2$$ and $$6 | q.$$
                
                Therefore both both $p$ and $q$ smaller natural
                numbers exist which can be found by dividing by
                $6$. The same must hold for the smaller numbers, ad
                infinitum. However this is impossible in the set of
                natural numbers. Thus $\sqrt{6}$ cannot be rational.
                
            \end{solution}

            \part[5] Try using the same approach to prove that $\sqrt{9}$
                  is irrational. Where does it fail?
            \begin{solution}
                $9 | p^2$ does not imply $9 | p$.
            \end{solution}

            \part[15] One can prove that $\sqrt{n}$ is irrational unless $n
                  = m^2$ for some positive integer $m$ (we say that
                  $n$ is a perfect square). Assuming this, prove that
                  $\sqrt{n_1} + \sqrt{n_2}$ is irrational unless both
                  $n_i$ are perfect squares.
            \begin{solution}
                Suppose that the sume is rational, and 
                $$\sqrt{n_1} + \sqrt{n_2} = \frac{p}{q}$$
                where $p$ and $q$ are relatively prime. Then 
                $$\sqrt{n_1} = \frac{p}{q} - \sqrt{n_2}$$
                $$q\sqrt{n_1} = p - q\sqrt{n_2}$$
                Suppose, without loss of generality, that $n_1$ is not
                a perfect square, but $n_2$ is, $n_2 = m^2.$
                $$q\sqrt{n_1} = p - qm.$$
                $$q^2n_1 = (p - qm)^2.$$
                Since the right side is a perfect square, and $q^2$
                is, then $n_1$ is a perfect square. Contradiction!
            \end{solution}
        \end{parts}

    % Niven/et. al. 1.2.5
    \question[4] How many integers between 100 and 1000 are divisible by
    7?
    \begin{solution}
        Note that $100 = 14\cdot 7 + 2$ and $1000 = 142\cdot 7 +
        6$. Thus numbers of the form $7n$ for $n = 15, \ldots
        142$. There are $142 - 14 = 128$ such $n$.
    \end{solution}

    % Niven/et. al. 1.2.9
    \question[3] Show that if $ac | bc$, then $a | b$.
    \begin{solution}
        The hypothesis implies that there is $m$ such that,
        $$bc = acm$$
        $$b = am$$
        i.e., $a | b$.
    \end{solution}

    % Niven/et. al. 1.2.10
    \question[3] Show that if $a | b$ and $c | d$ then $ac | bd$.
    \begin{solution}
        By hypothesis there is $m$ and $n$ such that
        $$b = am,\ d = cn$$
        $$bd = ac(mn)$$
        So $ac | bd.$
    \end{solution}

    % Niven/et. al. 1.2.11
    \question[8] Prove that $4 \nmid (n^2 + 2)$ for any integer $n$.
    \begin{solution}
        Using the division algorithm write $n = 4q + r$ where $0 \le r
        < 4.$ Then 
        $$n^2 + 2 = 16q^2 + 8qr + r^2 + 2$$
        $$ = 4(4q^2 + 2qr) + r^2 + 2.$$
        Examining possibilities for $r$ we see that the possibilities
        for $r^2 + 2$ are: 2, 3, 6 and 11. If $n^2 + 2$ were divisible
        by 4, then by the linear combination rule of divisbility, we'd
        have that $n^2 +2 - 4(4q^2 + 2qr) = r^2 + 2$ were divisible by
        4. However, by exhaustion 2, 3, 6, and 11 are not. 
    \end{solution}
\end{questions}

\section{Lecture 2}

\begin{questions}

% Existence of greatest common divisor


% Existence of gcd as a linear combination
% $(ma, mb) = m(a, b)$
% $(a, m) = 1 = (b, m)$ then $(ab, m) = 1$
% Definition of relatively prime.
% Adding some combination of $a$ to $b$ does not change the gcd of
%     $a$ and $b$.
% The Euclidean Algorithm

\end{questions}

\end{document}
