../../template.tex

\begin{document}

\printanswers

\title{575 Homework 1 Part II}
\maketitle

This section has \numpoints\ points.

\section{Lecture 2}

\begin{questions}

    % Niven/et. al. 1.2.5
    \question[4] How many integers between 100 and 1000 are divisible by
    7?
    \begin{solution}
        Note that $100 = 14\cdot 7 + 2$ and $1000 = 142\cdot 7 +
        6$. Thus numbers of the form $7n$ for $n = 15, \ldots
        142$. There are $142 - 14 = 128$ such $n$.
    \end{solution}

    % Niven/et. al. 1.2.9
    \question[3] Show that if $ac | bc$, then $a | b$.
    \begin{solution}
        The hypothesis implies that there is $m$ such that,
        $$bc = acm$$
        $$b = am$$
        i.e., $a | b$.
    \end{solution}

    % Niven/et. al. 1.2.10
    \question[3] Show that if $a | b$ and $c | d$ then $ac | bd$.
    \begin{solution}
        By hypothesis there is $m$ and $n$ such that
        $$b = am,\ d = cn$$
        $$bd = ac(mn)$$
        So $ac | bd.$
    \end{solution}

    % Niven/et. al. 1.2.11
    \question[8] Prove that $4 \nmid (n^2 + 2)$ for any integer $n$.
    \begin{solution}
        Using the division algorithm write $n = 4q + r$ where $0 \le r
        < 4.$ Then 
        $$n^2 + 2 = 16q^2 + 8qr + r^2 + 2$$
        $$ = 4(4q^2 + 2qr) + r^2 + 2.$$
        Examining possibilities for $r$ we see that the possibilities
        for $r^2 + 2$ are: 2, 3, 6 and 11. If $n^2 + 2$ were divisible
        by 4, then by the linear combination rule of divisbility, we'd
        have that $n^2 +2 - 4(4q^2 + 2qr) = r^2 + 2$ were divisible by
        4. However, by exhaustion 2, 3, 6, and 11 are not. 
    \end{solution}

    % Niven/et. al. 1.2.43
    \question[5] Prove that $a|bc$ if and only if $\frac{a}{(a, b)} |
    c$.

    \begin{solution}
      Assume $a|bc$. Then 
      $$bc = ak$$
      $$\frac{bc}{(a, b)} = \frac{ak}{(a, b)}$$
      Since $\frac{b}{(a, b)}$ is an integer, we have that
      $\frac{a}{(a, b)}$ divides $c$.

      Now assume that $\frac{a}{(a, b)} | c$. Then
      $$c = \frac{ak}{(a, b)}$$
      $$bc = \frac{b}{(a, b)} ak$$
      Since $\frac{b}{(a, b)}$ is an integer, we have that $a | bc$.
    \end{solution}

    % Niven et. al. 1.2.24
    \question[5] Prove that no integers $x, y$ exist satisfying $x + y
    = 100$ and $(x, y) = 3$.
    
    \begin{solution}
      Recall that by the addition property $(x, y) = (x + y, y)$. But
      by the hypothesis:
      $$3 = (x, y) = (100, y)$$
      which would imply 3 divides 100. Contradiction! Thus there are
      no such $x, y$.
    \end{solution}

    % Niven et. al. 1.2.23
    \question[3] Prove that the square of any integer is of the form
    $3k$ or $3k+1$, but not $3k+2$.

    \begin{solution}
      First note that $3^2 = 9$ and $2^2 = 4$ show that $3k$ and
      $3k+1$ are possible.

      Now suppose that $x^2 = 3k+2$. 

      There are three posibilites for $x$:
      $$(3k)^2 = 9k^2$$
      $$(3k+1)^2 = 9k^2 + 6k + 1 = 3(3k^2 + 2k) + 1$$
      $$(3k+2)^2 = 9k^2 + 12k + 4 = 3(3k^2 + 4k + 1) + 1$$

      Thus no square can be of the form $3k+2$.
    \end{solution}

\end{questions}

\end{document}
