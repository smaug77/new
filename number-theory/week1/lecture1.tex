% $Header: /cvsroot/latex-beamer/latex-beamer/solutions/generic-talks/generic-ornate-15min-45min.en.tex,v 1.5 2007/01/28 20:48:23 tantau Exp $

\documentclass{beamer}

% This file is a solution template for:

% - Giving a talk on some subject.
% - The talk is between 15min and 45min long.
% - Style is ornate.



% Copyright 2004 by Till Tantau <tantau@users.sourceforge.net>.
%
% In principle, this file can be redistributed and/or modified under
% the terms of the GNU Public License, version 2.
%
% However, this file is supposed to be a template to be modified
% for your own needs. For this reason, if you use this file as a
% template and not specifically distribute it as part of a another
% package/program, I grant the extra permission to freely copy and
% modify this file as you see fit and even to delete this copyright
% notice. 


\mode<presentation>
{
  \usetheme{Warsaw}
  % or ...

  \setbeamercovered{transparent}
  % or whatever (possibly just delete it)
}


\usepackage[english]{babel}
% or whatever

\usepackage[latin1]{inputenc}
% or whatever

\usepackage{times}
\usepackage[T1]{fontenc}
% Or whatever. Note that the encoding and the font should match. If T1
% does not look nice, try deleting the line with the fontenc.


\title[Division Algorithm] % (optional, use only with long paper titles)
{1.2: Induction and the Division Algorithm}

%\subtitle
%{Presentation Subtitle} % (optional)

%\author[Author, Another] % (optional, use only with lots of authors)
%{F.~Author\inst{1} \and S.~Another\inst{2}}
% - Use the \inst{?} command only if the authors have different
%   affiliation.

%\institute[Universities of Somewhere and Elsewhere] % (optional, but mostly needed)
%{
%  \inst{1}%
%  Department of Computer Science\\
%  University of Somewhere
%  \and
%  \inst{2}%
%  Department of Theoretical Philosophy\\
%  University of Elsewhere}
% - Use the \inst command only if there are several affiliations.
% - Keep it simple, no one is interested in your street address.

%\date[Short Occasion] % (optional)
%{Date / Occasion}

\subject{Talks}
% This is only inserted into the PDF information catalog. Can be left
% out. 



% If you have a file called "university-logo-filename.xxx", where xxx
% is a graphic format that can be processed by latex or pdflatex,
% resp., then you can add a logo as follows:

% \pgfdeclareimage[height=0.5cm]{university-logo}{university-logo-filename}
% \logo{\pgfuseimage{university-logo}}



% Delete this, if you do not want the table of contents to pop up at
% the beginning of each subsection:
\AtBeginSubsection[]
{
  \begin{frame}<beamer>{Outline}
    \tableofcontents[currentsection,currentsubsection]
  \end{frame}
}


% If you wish to uncover everything in a step-wise fashion, uncomment
% the following command: 

%\beamerdefaultoverlayspecification{<+->}


\begin{document}

\begin{frame}
  \titlepage
\end{frame}

\begin{frame}{Outline}
  \tableofcontents
  % You might wish to add the option [pausesections]
\end{frame}


% Since this a solution template for a generic talk, very little can
% be said about how it should be structured. However, the talk length
% of between 15min and 45min and the theme suggest that you stick to
% the following rules:  

% - Exactly two or three sections (other than the summary).
% - At *most* three subsections per section.
% - Talk about 30s to 2min per frame. So there should be between about
%   15 and 30 frames, all told.

\section{Induction}

\begin{frame}{Induction overview}
  % - A title should summarize the slide in an understandable fashion
  %   for anyone how does not follow everything on the slide itself.

  In induction we will cover:
  \begin{itemize}
  \item When to use induction
  \item What is induction
  \item Examples
  \end{itemize}
\end{frame}

\begin{frame}{When to use Induction}
  Induction:
  \begin{itemize}
  \item Used to prove incrementally.
  \item Useful when a fact is parameterizable by an integer.
  \item Often used to prove formula for sums.
  \item Can also be used to prove facts about integers of a certain
    form.
  \end{itemize}
\end{frame}

\begin{frame}{What is induction}
  \begin{itemize}
  \item Is a fundamental fact of the natural numbers.
  \item Statement: If $S$ is a set and 
    \begin{itemize}
    \item $0 \in S$
    \item $n \in S$ implies $n + 1 \in S$.
    \end{itemize}
    then $S$ is all natural numbers.
  \end{itemize}
\end{frame}

\begin{frame}{Induction Example}
  Prove $\sum_{k=0}^n k = {n+1 \choose 2}$
  \begin{itemize}
  \item for $n = 1$, $1 = {2 \choose 2} = 1$
  \item $$\sum_{k=0}^{n+1} k = \sum_{k=0}^n k + n+1$$
    $$= {n+1 \choose 2} + n+1$$
    $$=\frac{n^2 + 3n + 2}{2}$$
    $$=\frac{(n+2)(n+1)}{2}$$
    $$={n+2 \choose 2}.$$
  \end{itemize}
\end{frame}
    

\section{Divisibility}

\begin{frame}{Overview}
  \begin{itemize}
  \item Definition: What is divisibility?
  \item Some notational examples
  \item The division algorithm
  \end{itemize}
\end{frame}

\subsection{What is divisibility?}

\begin{frame}{Definition}
An integer $b$ is divisible by an integer $a$, not zero, if there is
an integer $x$ such that $b = ax$, and we write $a | b$. 
\end{frame}

\begin{frame}{Examples}
  \begin{itemize}
  \item $1 | x$ for all $x$
  \item $-1 | x$ for all $x$
  \item $x | 0$ for all $x \ne 0$.
  \item $a | \pm ab$ for all $b$ and $a \ne 0$.
  \end{itemize}
\end{frame}

\begin{frame}{Proper Divisor}
  If $a|b$ and $0 < a < b$ then $a$ is a proper divisor of $b$.
  \begin{itemize}
  \item $12$ has $1, 2, 3, 4, 6$ as proper divisors.
  \item $1$ has no proper divisors.
  \end{itemize}
\end{frame}

\subsection{Some Notational Examples}

\begin{frame}{Arbitrary constants}

  $$a|b \implies a|bc \mbox{ for all integer } c.$$
  
  Thus if $a$ divides $b$, we can add whatever factors we want to $b$
  and $a$ will still divide it.

  $$3 | 6 \implies 3|12, 3|18, 3|24, \ldots$$
\end{frame}

\begin{frame}{Proof}
  
  $$a|b \implies b = ax$$
  $$cb = cax = (cx)a$$
  $$a | bc$$

\end{frame}

\begin{frame}{Transitivity}
  $$a|b, b|c \implies a|c$$
  
  This is the often known transitivitity.

  \begin{itemize}
  \item Note $b \ne 0$ (since 0 divides nothing).
  \item Otherwise, since $x|0$ for all $0$, we could have $x|0$, $0|y$
    and thus $x|y$ for all $x$ and $y$.
  \end{itemize}
\end{frame}

\begin{frame}{Proof}
  $$a|b, b|c \implies b = ax, c = by$$
  $$c = by = axy = a(xy)$$
  $$a | c$$
\end{frame}

\begin{frame}{Linear Combinations}

  $$a|b, a|c \implies a|(bn+cm)$$

  Thus $a$ divides the lattice generated by $b$ and $c$. Hence we can
  picture this with some geometry if we wanted.
\end{frame}

\begin{frame}{Proof}
  $$a|b, a|c \implies b = ax, c = ay$$
  $$bn = a(xn), cm = a(ym)$$
  $$bn+cm = a(xn+ym)$$
  $$a | (bn + cm)$$
\end{frame}

\begin{frame}{Trivialities}

  $$a|b, b|a \implies a = \pm b$$
  $$a|b, a> 0, b>0, \implies a \le b$$
  
\end{frame}

\begin{frame}{Proof}
  $$a|b \implies b = ax$$
  $$b|a \implies a = by$$
  $$a = by = axy$$
  $$1 = xy$$
  Since $x$ $y$ are integers, then $x, y = \pm 1$.
\end{frame}
  
\begin{frame}{Multiplicative constants}
  $$m \ne 0, a|b \implies ma | mb$$
  
  Thus by adding the same factors to both sides, we preserve
  divisibility. (Golden Rule?)

\end{frame}

\begin{frame}{Proof}
  $$a|b \implies b = ax$$
  $$mb =max$$
  $$ma | mb$$
\end{frame}

\subsection{The Division Algorithm}

\begin{frame}{Statement}

Given integers $a$, $b$, with $a > 0$, there exist unique integers $q$
and $r$ such that $b = qa + r$, $0 \le r < a$.

$$25 = 7 \cdot 3 + 4$$
\end{frame}

\begin{frame}
Notice that only certain $r$s are possible, and they are unique.

$$a = 4, r = 0, 1, \ldots, 3$$

\begin{center}
  \begin{tabular}{c|c}
    b & r \\
    \hline
    0 & 0 \\
    1 & 1 \\
    2 & 2 \\
    3 & 3 \\
    4 & 0 \\
    5 & 1 \\
    6 & 2 \\
  \end{tabular}
\end{center}
\end{frame}

\end{document}


