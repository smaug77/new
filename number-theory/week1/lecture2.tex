% $Header: /cvsroot/latex-beamer/latex-beamer/solutions/generic-talks/generic-ornate-15min-45min.en.tex,v 1.5 2007/01/28 20:48:23 tantau Exp $

\documentclass{beamer}

% This file is a solution template for:

% - Giving a talk on some subject.
% - The talk is between 15min and 45min long.
% - Style is ornate.



% Copyright 2004 by Till Tantau <tantau@users.sourceforge.net>.
%
% In principle, this file can be redistributed and/or modified under
% the terms of the GNU Public License, version 2.
%
% However, this file is supposed to be a template to be modified
% for your own needs. For this reason, if you use this file as a
% template and not specifically distribute it as part of a another
% package/program, I grant the extra permission to freely copy and
% modify this file as you see fit and even to delete this copyright
% notice. 


\mode<presentation>
{
  \usetheme{Warsaw}
  % or ...

  \setbeamercovered{transparent}
  % or whatever (possibly just delete it)
}


\usepackage[english]{babel}
% or whatever

\usepackage[latin1]{inputenc}
% or whatever

\usepackage{times}
\usepackage[T1]{fontenc}
% Or whatever. Note that the encoding and the font should match. If T1
% does not look nice, try deleting the line with the fontenc.


\title[GCDs] % (optional, use only with long paper titles)
{1.2: GCDs and their properties}

%\subtitle
%{Presentation Subtitle} % (optional)

%\author[Author, Another] % (optional, use only with lots of authors)
%{F.~Author\inst{1} \and S.~Another\inst{2}}
% - Use the \inst{?} command only if the authors have different
%   affiliation.

%\institute[Universities of Somewhere and Elsewhere] % (optional, but mostly needed)
%{
%  \inst{1}%
%  Department of Computer Science\\
%  University of Somewhere
%  \and
%  \inst{2}%
%  Department of Theoretical Philosophy\\
%  University of Elsewhere}
% - Use the \inst command only if there are several affiliations.
% - Keep it simple, no one is interested in your street address.

%\date[Short Occasion] % (optional)
%{Date / Occasion}

\subject{Talks}
% This is only inserted into the PDF information catalog. Can be left
% out. 



% If you have a file called "university-logo-filename.xxx", where xxx
% is a graphic format that can be processed by latex or pdflatex,
% resp., then you can add a logo as follows:

% \pgfdeclareimage[height=0.5cm]{university-logo}{university-logo-filename}
% \logo{\pgfuseimage{university-logo}}



% Delete this, if you do not want the table of contents to pop up at
% the beginning of each subsection:
\AtBeginSubsection[]
{
  \begin{frame}<beamer>{Outline}
    \tableofcontents[currentsection,currentsubsection]
  \end{frame}
}


% If you wish to uncover everything in a step-wise fashion, uncomment
% the following command: 

%\beamerdefaultoverlayspecification{<+->}


\begin{document}

\begin{frame}
  \titlepage
\end{frame}

\begin{frame}{Outline}
  \tableofcontents
  % You might wish to add the option [pausesections]
\end{frame}


% Since this a solution template for a generic talk, very little can
% be said about how it should be structured. However, the talk length
% of between 15min and 45min and the theme suggest that you stick to
% the following rules:  

% - Exactly two or three sections (other than the summary).
% - At *most* three subsections per section.
% - Talk about 30s to 2min per frame. So there should be between about
%   15 and 30 frames, all told.

\section{Proof of the Division Algorithm}

\begin{frame}{Recall the Division Algorithm}
  % - A title should summarize the slide in an understandable fashion
  %   for anyone how does not follow everything on the slide itself.

  \begin{itemize}
  \item Given integers $a$, $b$, with $a > 0$, there exist unique integers $q$
    and $r$ such that $b = qa + r$, $0 \le r < a$.

  \item  We will prove this theorem, and then use it to prove a fundamental
  fact about GCDs in the next section.

  \item  We first prove that there is such an $r$ and $q$, then prove
    $r$ and $q$ are unique.

  \end{itemize}

\end{frame}

\begin{frame}{Proof: existence}
  \begin{itemize}
  \item Consider set of all $b \pm ka$.
  \item Well-ordering $\implies$ there is a smallest element.
  \item $r = $ this smallest element.
  \item $r = b -qa$
  \end{itemize}
\end{frame}

\begin{frame}{Proof: uniqueness}
  Suppose $q_1$, $r_1$ is another pair.
  \begin{itemize}
  \item $r < r_1$ by choice of $r$.
  \item $0 < r - r_1  = a(q - q_1) < a$ (since $r < a$)
  \item Thus $a | r - r_1$. So $a$ divides a number smaller than it in
    absolute value.
  \item This contradicts a fact about $|$ from lecture 1.
  \item $r = r_1$. Hence $q = q_1$.
  \end{itemize}
\end{frame}


\section{Greatest Common Divisor}

\subsection{Definition, Existence}

\begin{frame}{Definition}

  A {\it common divisor} of two numbers $b$, $c$ is an integer $a$
  such that $a | b$ and $a | c$.

  \begin{itemize}
  \item There are a finite number of divisors of any non-zero integer.
    \pause
  \item Because if $a | c$ then $-c \le a \le c$.
    \pause
  \item Thus there are a finite number of common divisors.
    \pause
  \item Unless $b = c = 0$.
  \end{itemize}

\end{frame}

\begin{frame}{Examples}
  \begin{itemize}
  \item $b = 14, c = 21$: $-7, -1, 1, 7$
  \item $b = 36, c = 54$: $\pm 18, \pm 9, \pm 6, \pm 3, \pm 2, \pm 1$ 
  \item $b = 1, c = 14$: $\pm 1$
  \item $b = 0, c = 14$: $\pm 14, \pm 7, \pm 2, \pm 1$
  \item $b = 14, c = 14$: $\pm 14, \pm 7, \pm 2, \pm 1$
  \end{itemize}
\end{frame}

\begin{frame}{linear combinations}

  Recall from lecture 1: If $a | b$ and $a | c$ then $a|(x_0b +
  y_0c)$.
  \begin{itemize}
  \item Thus a common divisor also divides $\mathbb{Z}-$linear
    combinations.
  \end{itemize}
\end{frame}
  
\begin{frame}{Greatest common divisors}
  Since there are a finite number of common divisors, there is a
  greatest one.
  \begin{itemize}
  \item Note well-ordering again.
  \item Of course well-ordering = induction.
  \end{itemize}
\end{frame}

\begin{frame}{Examples}
  \begin{itemize}
  \item $(14, 21) = 7$
  \item $(36, 54) = 18$
  \item $(1, 14) = 1$
  \item $(0, 14) = 14$
  \item $(14, 14) = 14$
  \end{itemize}
  GCD is always positive!
\end{frame}

\subsection{M\'eziriac-B\'ezout Identity}

\begin{frame}{The Identity}

  If $g$ is the common divisor of $b$ and $c$, then there exist $x_0,
  y_0$ such that $$g = (b, c) = bx_0 + cy_0.$$

  \begin{itemize}
  \item First known statement is M\'eziriac in the 1600s
  \item Most often called B\'ezout identity, but he proved it for
    polynomials.
  \item The gcd is expressible as a $\mathbb{Z}$-linear combination of
    the two integers.
  \item Recall that any common divisor divides $\mathbb{Z}$-linear
    combinations.x1
  \item So this is a sort of converse.
  \end{itemize}

\end{frame}

\begin{frame}{Proof Outline}
  
  \begin{enumerate}
  \item Choose the smallest (positive) $\mathbb{Z}$-linear
    combination, $l$.
  \item Prove that $l$ is a common divisor using the division
    algorithm and proof by contradiction (using the choice of $l$).
  \item Note that a common divisor divides any $\mathbb{Z}$-linear
    combination, thus $g | l$.  Conclude the theorem.
  \end{enumerate}

\end{frame}

\begin{frame}{Step 2}

  (Step 1 and 3 being easy).
  
  \begin{itemize}
  \item Without loss of generality, only prove $l | b$.
  \item Assume $l$ does not divide $b$.
  \item Division algorithm gives $r > 0$ such that $r = b - lq$.
  \item $b - lq = b(1 - qx_0) + c(-qy_0)$ so $r$ is in the set $l$ is
    chosen from.
  \item Thus $r > l$. This contradicts the choice of $r$ from the
    division algorithm.
  \end{itemize}
\end{frame}

\begin{frame}{Consequence}

  We could have defined GCD this way:
  \begin{theorem} The greatest common divisor of $b$ and $c$ is the
    least positive $\mathbb{Z}$-linear combination of $b$ and $c$.
  \end{theorem}
  
  Or this way:
  \begin{theorem} The greatest common divisor of $b$ and $c$ is the
    positive common divisor  that is divisible by every other common
    divisor.
  \end{theorem}

\end{frame}

\subsection{How dividing and multiplying effects GCDs}

\begin{frame}{Common factors}

  \begin{theorem} For any positive integer $m$, 
    $$(ma, mb) = m(a, b)$$
  \end{theorem}
  
  \begin{proof}
    $(ma, mb)$ is the least positive value of $max+mby$, which is the
    same as the least positive value of $ax+by$ times $m$.
  \end{proof}
  
\end{frame}

\begin{frame}{Common factors}

  \begin{theorem}
    If $d | a$ and $d | b$ and $d > 0$ then 
    $$(\frac{a}{d}, \frac{b}{d}) = \frac{1}{d}(a, b)$$
  \end{theorem}

This is just a restatement of the previous statement.

\end{frame}

\begin{frame}{What do these common factor statements mean?}

  What we are really saying is, by definition:
  $$g = (a, b)$$
  
  $$( \frac{a}{g}, \frac{b}{g} ) = 1$$

  i.e. dividing doesn't do something non-atomic.

\end{frame}

\begin{frame}{Relatively prime pairs}

  If both $a$ and $b$ are relatively prime to $m$, so is $ab$.
  
  \begin{proof}
    By the M\'eziriac-B\'ezout identity 
    $$1 = ax_0 + my_0 = bx_1 + my_1$$
    $$1 = 1 \cdot 1 = abx_0x_1 + m( Stuff)$$
    i.e. by the identity again $ab, m) = 1$.
  \end{proof}
  
  Thus, multiplying doesn't create new factors.

\end{frame}

\begin{frame}{relatively prime}

  Note: We used the term relatively prime. Above. It's defined how you
  think.

\end{frame}

\begin{frame}{One must fall!}

  If $b$ and $c$ are relatively prime, and $c | ab$, then $c|a$.

\end{frame}

\subsection{How addition effects GCDs}

\begin{frame}{general addition}

In general addition is screwy:

$$(12, 2) = 2$$
$$(12, 3) = 3$$
$$(12, 4) = 4$$
$$(12, 5) = 1$$

\end{frame}

\begin{frame}{Adding multiples}

But ...
\begin{theorem}
  $$d = (a, b) = (a, b + ax)$$
\end{theorem}

This isn't as satisfying as the results on division/multiplication,
but next lecture we will see it is very powerful.

\end{frame}

\begin{frame}{Proof Outline}
Let,
  $$d = (a, b) $$
  $$g = (a, b + ax)$$
  
  \begin{enumerate}
  \item Show that $d | b + ax$.
  \item Show $g | d$.
  \item Since $d | b + ax$, by lecture 1 $d | g$.
  \item Conclude $d = \pm g$.
  \item Since $d, g > 0$ $d = g$.
  \end{enumerate}
    
\end{frame}

\begin{frame}{Show $d | b + ax$}
  Since $d | a$ and $d | b$ by definition, we have by the linear
  combination property from lecture 1 that $d | b + ax$.
\end{frame}

\begin{frame}{Show $g | d$.}
  By the  M\'eziriac-B\'ezout identity there are $x_0$ and $y_0$:
  $$d = ax_0 + by_0$$
  $$d = a(x_0 - xy_0) + (b + ax)y_0$$
  Since $d$ is a linear combination of $a$ and $(b+ax)$ by the
  definition of $g$ and the linear combination property from lecture
  1, we have $g | d$.
\end{frame}

\end{document}


