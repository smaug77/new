\documentclass{exam}
\usepackage{amssymb} % for \nmid

\title{Quiz 1}


\begin{document}
\maketitle

\begin{questions}
  \question[5] State the division algorithm

  \question[5] Prove that, for all positive integers $n$,
  $$1^3 + 2^3 + \cdots + n^3 = \frac{n^2(n+1)^2}{4}.$$
  \begin{solution}
    Base case: $1^3 = \frac{1^2(2)^2}{4} = 1$.

    Induction Step: 
    $$\sum_{k=1}^{n+1} k^3 = (n+1)^3 + \sum_{k=1}^n k^3$$
    $$  (n+1)^3 + \frac{n^2(n+1)^2}{4}
    = \frac{ 4(n+1)^3 + n^2(n+1)^2 }{4}$$
    $$ = \frac{ (n+1)^2 ( 4(n+1) + n^2 ) }{4}
    = \frac{ (n+1)^2 (n+2)^2 }{4}.$$        
  \end{solution}


  \question[5]  State the M\'eziriac-B\'ezout identity.

    % Niven/et. al. 1.2.11
    \question[8] Prove that $4 \nmid (n^2 + 2)$ for any integer $n$.
    \begin{solution}
        Using the division algorithm write $n = 4q + r$ where $0 \le r
        < 4.$ Then 
        $$n^2 + 2 = 16q^2 + 8qr + r^2 + 2$$
        $$ = 4(4q^2 + 2qr) + r^2 + 2.$$
        Examining possibilities for $r$ we see that the possibilities
        for $r^2 + 2$ are: 2, 3, 6 and 11. If $n^2 + 2$ were divisible
        by 4, then by the linear combination rule of divisbility, we'd
        have that $n^2 +2 - 4(4q^2 + 2qr) = r^2 + 2$ were divisible by
        4. However, by exhaustion 2, 3, 6, and 11 are not. 
    \end{solution}


\end{questions}

\end{document}